\documentclass[11pt]{article}
\usepackage[margin=1.6in]{geometry}
\usepackage[utf8]{inputenc}
\usepackage[T1]{fontenc}
\usepackage{fixltx2e}
\usepackage{graphicx}
\usepackage{longtable}
\usepackage{float}
\usepackage{wrapfig}
\usepackage{amsmath, amsthm}
\usepackage{textcomp}
\usepackage{marvosym}
\usepackage{wasysym}
\usepackage{amssymb}
\usepackage[frenchb]{babel}
\usepackage{color}
\usepackage{listings}
\lstset{
  %frame=tb,
  language=Bash,
  aboveskip=2mm,
  belowskip=2mm,
  showstringspaces=false,
  columns=flexible,
  basicstyle={\small\ttfamily},
  numbers=none,
  numberstyle=\footnotesize\color{gray},
  %keywordstyle=\color{blue},
  %commentstyle=\color{dkgreen},
  %stringstyle=\color{mauve},
  breaklines=true,
  breakatwhitespace=true,
  tabsize=3
}
         
% Pour XeTeX
\XeTeXdefaultencoding utf-8
\usepackage{fontspec}

\newenvironment{absolutelynopagebreak}
  {\par\nobreak\vfil\penalty0\vfilneg
   \vtop\bgroup}
  {\par\xdef\tpd{\the\prevdepth}\egroup
   \prevdepth=\tpd}


\definecolor{dkgreen}{rgb}{0,0.6,0}
\definecolor{gray}{rgb}{0.5,0.5,0.5}
\definecolor{mauve}{rgb}{0.58,0,0.82}

\theoremstyle{definition}
\newtheorem*{myRem}{Remarque}

\tolerance=1000
\setcounter{secnumdepth}{2}
\author{Borne, Duquennoy, Duverney, Isnel}
\date{}
\title{Nachos Pré-rapport}

\begin{document}
\maketitle



\section{Fonctionnalités}
Environnement utilisateur multi-processus avec espaces d'adressages séparés et virtualisation mémoire.
Processus utilisateur multi-threadés avec primitives de synchronisation (Mutex, Semaphores, variables-conditions). Gestion des entrées-sorties (GetInt, PutInt, GetChar, PutChar, GetString, Putstring).
Système de fichier. Protocole réseau sans connexion avec envoi fiable de données.
Utilisation interactive via un Shell (exec, cd, ls, mkdir, open ...).
\section{Spécifications}
\subsection{Entrées-Sorties}

\subsubsection{\texttt{void PutChar(char c)}}
\begin{itemize}
\item[-] Sémantique: Écrit le caractère \texttt{c} sur la sortie standard.
\end{itemize}

\subsubsection{\texttt{void PutString(const char *s);}}
\begin{itemize}
\item[-] Sémantique: Écrit la chaîne de caractères lue à l'adresse \texttt{s}.
\end{itemize}

\subsubsection{\texttt{char GetChar()}}
\begin{itemize}
\item[-] Sémantique: Lit un caractère depuis l'entrée standard et retourne le caractère lu.
\end{itemize}

\subsubsection{\texttt{void GetString(char *s, int n)}}
\begin{itemize}
\item[-] Sémantique:
  Lit une chaîne de caractères de longueur maximale égale à \texttt{n} depuis l'entrée standard et
  l'écrit à l'adresse \texttt{s}.
\item[-] Préconditions: L'adresse \texttt{s} est valide (espace suffisant).
\end{itemize}

\subsubsection{void PutInt(int n)}
\begin{itemize}
\item[-] Sémantique: Écrit l'entier \texttt{n} sur la sortie standard.
\end{itemize}

\subsubsection{\texttt{void GetInt(int *n)}}
\begin{itemize}
\item[-] Sémantique: Lit un entier depuis l'entrée standard et l'écrit à l'adresse \texttt{n}.
\item[-] Préconditions: L'adresse \texttt{n} est valide.
\end{itemize}

\subsection{Processus et Threads}

\subsubsection{\texttt{int ForkExec(char * fileName)}}
 \begin{itemize}
 \item[-] Sémantique: Crée un nouveau processus qui exécute le fichier dont le nom est fourni en paramètre
\item[-] Préconditions: "fileName" est le nom d'un fichier exécutable au format \texttt{noff}.
 \end{itemize}

\subsubsection{\texttt{int UserThreadCreate(void f(void* arg), void* arg)}}
\begin{itemize}
\item[-] Spécifications: Prends en paramètres un pointeur de fonction "f" ne retournant pas de valeur
  et un pointeur "arg".
\item[-] Sémantique: Crée un nouveau thread utilisateur qui exécute la fonction \texttt{f(arg)}.
  f est un pointeur de fonction 
\item[-] Préconditions: Le système doit disposer d'une quantité de mémoire suffisante pour allouer la
  pile du thread à créer. 
\item[-] Valeur de retour: retourne l'identficateur du thread crée, $-1$ si une erreur s'est produite lors de
  la création du thread.
\end{itemize}

\subsubsection{\texttt{void UserThreadExit()}}
\begin{itemize}
\item[-] Sémantique: Termine l'exécution du thread courant.
\end{itemize}

\subsubsection{\texttt{int UserThreadJoin(int tid)}}
\begin{itemize}
\item[-] Sémantique: Attends la terminaison du thread d'identifiant "tid", renvoie $-1$
  si le thread est déjà terminé, $0$ sinon.
\item[-] Préconditions: "tid" est un identifiant de thread valide, initialisé dans le processus courant
  à l'aide d'un appel à \texttt{UserThreadCreate}. Le thread courant n'a pas déjà fait \texttt{join(tid)}.
\end{itemize}

\subsection{Synchronisation}
\subsubsection{\texttt{Mutex\_t MutexCreate()}}
\begin{itemize}
\item[-]Sémantique: Initialise un mutex.
\item[-]Valeur de retour: Un identifiant de type \texttt{Mutex\_t} pour le mutex.
\end{itemize}

\subsubsection{\texttt{void MutexLock(Mutex\_t mutexId)}}
\begin{itemize}
\item[-]Sémantique: Acquiert le mutex d'identifiant "mutexId". Si "mutexId" est déverrouillé, il devient verrouillé
  et possédé par le thread appelant. Si le mutex d'identifiant "mutexId" est déjà verrouillé par un autre thread,
  le thread appelant est suspendu jusqu'à ce que "mutexId" soit déverrouillé.
\item[-]Pré-Condition: "mutexId" est un identifiant de mutex valide, initialisé dans le processus
  appellant par la méthode \texttt{MutexCreate}.
\end{itemize}

\subsubsection{\texttt{void MutexUnlock(Mutex\_t mutexId)}}
\begin{itemize}
\item[-]Sémantique: Relâche le mutex d'identifiant "mutexId".
\item[-]Pré-Condition: "mutexId" est un identifiant de mutex valide, initialisé dans le processus
  courant par la méthode \texttt{MutexCreate}. "mutexId" est Verrouillé.
\end{itemize}

\subsubsection{\texttt{MutexDestroy(Mutex\_t mutexId)}}
\begin{itemize}
\item[-]Sémantique: Détruit le mutex "mutexId".
\item[-]Préconditions: "mutexId" est un identifiant de mutex valide, initialisé dans le processus
  courant par la méthode \texttt{MutexCreate}.
  Le verrou est relâché. Détruire un verrou non relâché mène à un comportement
  non déterminé.
\end{itemize}

\subsubsection{\texttt{Sem\_t SemCreate(int initialValue)}}
\begin{itemize}
\item[-]Sémantique: Initialise une sémaphore avec la valeur "initialValue".
\item[-]Valeur de retour: Un identifiant de type \texttt{Sem\_t} pour la variable-condition.
\end{itemize}

\subsubsection{\texttt{void SemWait(Sem\_t semaphoreId)}}
\begin{itemize}
\item[-]Sémantique: Le thread appelant attends que la sémaphore ait une valeur $>0$ et la décrémente. 
\item[-]Pré-Condition: "semaphoreId" est un identifiant de semaphore valide, initialisé dans le processus
  appellant par la méthode \texttt{SemCreate}.
\end{itemize}

\subsubsection{\texttt{void SemPost(Sem\_t semaphoreId)}}
\begin{itemize}
\item[-]Sémantique: Incrémente la valeur de la sémaphore, réveille un thread en attente de cette sémaphore
  si besoin.
\item[-]Pré-Condition: "semaphoreId" est un identifiant de semaphore valide, initialisé dans le processus
  courant par la méthode \texttt{SemCreate}.
\end{itemize}

\subsubsection{\texttt{void SemDestroy(Sem\_t semaphoreId)}}
\begin{itemize}
\item[-]Sémantique: Libère les ressources associées à la sémaphore.
\item[-]Pré-Condition: "semaphoreId" est un identifiant de semaphore valide, initialisé dans le processus
  courant par la méthode \texttt{SemCreate}.
\end{itemize}

\subsubsection{\texttt{Cond\_t CondCreate()}}
\begin{itemize}
\item[-]Sémantique: Initialise une variable-condition.
\item[-]Valeur de retour: Un identifiant de type \texttt{Cond\_t} pour la variable-condition.
\end{itemize}

\subsubsection{\texttt{void CondWait(Cond\_t condId, Mutex\_t mutedId)}}
\begin{itemize}
\item[-]Sémantique: Met le thread courant en sommeil dans la file d'attente associée à "condId",
  relâche le verrou "mutexId".
\item[-]Préconditions: "mutexId" et "condId" sont des identifiants de mutex et variables-conditions
  valides, initialisés dans le processus courant par les méthodes \texttt{MutexCreate} et \texttt{CondCreate}.
  "mutexId" est verrouillé.
\end{itemize}

\subsubsection{\texttt{void CondSignal(Cond\_t condId)}}
\begin{itemize}
\item[-]Sémantique: Réveille un thread en sommeil dans la file d'attente associée à la variable-condition
  "condId". Si aucun thread n'est présent dans la liste, le signal est perdu.
\item[-]Pré-Condition: "contId" est un identifiant de variable-condition valide, initialisé dans le processus
  courant par la méthode \texttt{CondCreate}.
\end{itemize}

\subsubsection{\texttt{void CondBroadCast(Cond\_t condId)}}
\begin{itemize}
\item[-]Sémantique: Réveille tous les threads en sommeil dans le file d'attente associée à la
  variable-condition "condId".
\item[-]Pré-Condition: "contId" est un identifiant de variable-condition valide, initialisé dans le processus
  courant par la méthode \texttt{CondCreate}.
\end{itemize}

\subsubsection{\texttt{void CondDestroy(Cond\_t condId)}}
\begin{itemize}
\item[-]Sémantique: Libère les ressources associées à la variable condition "condId". 
\item[-]Pré-Condition: "contId" est un identifiant de variable-condition valide, initialisé dans le processus
  courant par la méthode \texttt{CondCreate}. Aucun thread n'est en attente dans la file associée à la variable.
\end{itemize}

\subsection{Système de fichier}

\subsubsection{\texttt{void Create (char *name, int initialSize)}}
\begin{itemize}
\item[-]Sémantique: Crée un fichier de nom "name" et de taille "initialSize".
\end{itemize}

\subsubsection{\texttt{OpenFileId Open(char *name)}}
\begin{itemize}
\item[-]Sémantique: Ouvre le fichier dont le nom "name". 
\item[-]Valeur de retour: Retourne un descripteur de fichier de type \texttt{OpenFileId}
  permettant de lire et écrire dans le fichier ou $-1$ si l'ouverture à échoué.
\end{itemize}

\subsubsection{\texttt{void Write (char *buffer, int size, OpenFileId id)}}
\begin{itemize}
\item[-]Sémantique: Écrit "size" octets depuis le fichier dont le descripteur est "id"
  dans le buffer "buffer".
\item[-]Pré-condition: Le descripteur "id" doit être valide (fichier ouvert), initialisé dans le thread courant
  par la méthode \texttt{Open}. Tenter d'écrire dans un fichier non initialisé par open resulte en un comportement non spécifié.
\end{itemize}

\subsubsection{\texttt{int Read(char *buffer, int size, OpenFileId id)}}
\begin{itemize}
\item[-]Sémantique: Lit "size" octets depuis le fichier dont le descripteur est "id" dans le buffer
  "buffer". Si le fichier contient moins de "size" octets on lit tout les octets disponibles.
\item[-]Valeur de retour: Nombre d'octets lus.
\item[-]Pre-condition: Le descripteur doit être valide (fichier ouvert), initialisé dans le thread appelant
  par la méthode Open.
  Tenter de lire dans un fichier non initialisé par open résulte en un comportement non spécifié.
\end{itemize}

\subsubsection{\texttt{void Close(OpenFileId id)}}
\begin{itemize}
\item[-]Sémantique: Ferme le ficher dont le descripteur est "id".
\item[-]Pré-condition: Le descripteur doit être valide (fichier ouvert), initialisé dans le thread appelant
  par la méthode Open. Tenter d'écrire dans un fichier non initialisé par open résulte en un comportement non spécifié.
\end{itemize}

\subsubsection{\texttt{void CreateDirectory(char * name)}}
\begin{itemize}
\item[-]Sémantique: Créer un répertoire dans le système de fichier Nachos, de nom "name"
  passé en paramètre.
\end{itemize}

\subsubsection{\texttt{void ChangeDirectoryPath(char * name)}}
\begin{itemize}
\item[-]Sémantique: Change le répertoire courant vers le répertoire de nom "name".
\item[-]Exemple: \texttt{ChangeDirectoryPath("./Dossier1/Dossier2")}
\end{itemize}

\subsubsection{\texttt{void ListDirectory(char * name)}}
\begin{itemize}
\item[-]Sémantique: Liste tout les fichiers et documents du répertoire dont le nom est passé en paramètre.
\item[-]Exemple: \texttt{ListDirectory("./Dossier1/Dossier2")}
\end{itemize}

 \subsubsection{\texttt{int Remove(char * name)}}
\begin{itemize}
\item[-]Sémantique: Supprime le fichier ou répertoire passé en paramètre
\item[-]Valeur de retour: 1 si la suppression a réussi, 0 sinon.
\item[-]Exemple: \texttt{Remove("./Dossier1/Dossier2")}
\end{itemize}

\subsection{Réseau}

\subsubsection{\texttt{void SendMessage(int addressDesti, int boxTo, int boxFrom, char * data)}}
\begin{itemize}
\item[-]Sémantique: Envoi du message "data" depuis la boite
 "boxfrom" vers la machine d'adresse "addressDesti" dans la boite "boxTo".
\item[-]Pré\_condition: La machine "addressDesti" doit être prête à recevoir des messages,
  i.e. avoir exécuté \texttt{ReceiveMessage}, les numéros de boites "boxTo" et "boxFrom" sont compris entre $0$ et $9$.
\end{itemize}

\subsubsection{\texttt{void ReceiveMessage(char * data, int box)}}
\begin{itemize}
\item[-]Sémantique: Initialise la réception d'un message depuis la boite "box".
  Les données reçues sont stockées à l'adresse "data".
\item[-]Pré\_condition: Le numéro de boite "box".
\end{itemize}


\section{Implémentation}

\subsection{Processus et Threads}

\begin{lstlisting}
Fichiers: userprog/userthread.cc, userprog/userprocess.cc, userprog/addrspace.cc, thread/thread.cc
\end{lstlisting}
Nous modélisons un processus par un objet \texttt{AddrSpace}.
Les fonctionnalités rendues par un objet \texttt{AddrSpace} dépassent la simple gestion de la mémoire puisque l'on trouve encapsulé au sein de cette classe, les méthodes relatives, entre autres, à la restauration et la sauvegarde de l'état du processeur de la machine \texttt{MIPS}.
Partant de ce constat nous avons décidé d'étendre encore la sémantique de l'objet \texttt{addrSpace} en ajoutant une liste des threads actifs dans l'espace d'adressage, ainsi que les informations relatives aux éventuels appels à la méthode \texttt{join} entre ces threads.

Nous parlerons par la suite d'espace d'adressage au sens élargi décrit ci-dessus:
Une entité décrivant à la fois un espace mémoire, l'état de la machine \texttt{MIPS} et les informations relatives aux fils d'exécution actifs dans cet espace. Afin de faire le lien entre les différents threads d'un même processus/espace d'adressage nous donnons à chaque objet \texttt{thread} une référence vers un objet \texttt{addrspace}.

\begin{myRem}
Lors de notre réflexion sur l'implémentation des processus utilisateur, il nous semblait plus cohérent de déléguer à \texttt{addrspace} les méthodes de gestion de la mémoire et de créer une classe \texttt{processus} regroupant d'un côté un espace d'adressage et de l'autre les informations relatives aux fil d'exécutions vivant dans cet espace. Malheureusement en faisant ce choix nous nous somme trouvé face à des difficultés dans la re-factorisation du code nachos original et pour un petit gain de cohérence avons introduit une trop grande complexité de mise en oeuvre. Nous avons abandonné ce modèle au profit de celui décrit plus haut.
\end{myRem}

\subsubsection{Piles des threads}
Afin d'accueillir les piles de nouveaux threads on divise un espace d'adressage \texttt{addrSpace} en blocs de pages, de taille \texttt{NumPagesPerStack}.
Dans chaque espace, une bitmap \texttt{stackBitmap} mémorise l'état courant des emplacement de blocs de pages libres pour les piles. 

\subsubsection{Thread Join}
Au sein d'un processus, un thread peut utiliser la méthode join pour attendre la terminaison d'un autre thread.
On maintient dans chaque \texttt{addressSpace} une structure \texttt{joinMap}.
La Map associe à un \texttt{tid} une liste de Threads.

Lorsqu'un Thread $t1$ fait un \texttt{join} sur un Thread $t2$ on ajoute dans \texttt{joinMap}
l'association $(t2, [t1])$. Cette association signifie que $t2$ est attendu par $t1$.
Le Thread $t1$ se met alors en sommeil et attends $t2$. 
Si $t3$ fait un join sur $t2$ la \texttt{joinMap} est dans l'état $(t2, [t1,t3])$ et $t3$ est mis en sommeil.
Lorsque $t2$ termine son exécution il place $t1$ et $t3$ dans la \texttt{readylist} du scheduler.

On maintient une liste \texttt{threadList} des threads actifs dans un espace d'adressage.
Un thread $t1$ ne peut faire un join que vers un thread $t2$ actif.
Un thread peut être attendu par plusieurs threads distincts.

\subsubsection{Création d'un thread}
Toute création de thread utilisateur s'effectue par l'appel système UserTHreadcreate
qui déclenche un passage en mode noyau via l'instruction \texttt{syscal}.
Le gestionnaire d'exception implémenté dans la méthode \texttt{ExceptionHandler} de la classe
\texttt{exception.cc} appelle alors la méthode \texttt{do\_UserThreadCreate()}.
L'initialisation du nouveau fil d'éxecution s'effectue en trois étapes.

\begin{enumerate}
\item Dans \texttt{do\_UserThreadCreate()}:

On crée un nouvel objet thread $t$ (thread noyau propulseur), destiné à configurer et brancher la machine \texttt{MIPS} sur l'exécution d'une nouvelle fonction $f$ au sein de l'espace d'adressage du thread/processus courant.
On commence par récupérer la fonction utilisateur $f$ à exécuter et ses arguments $arg$ dans les registres $4$ et $5$ de la machine \texttt{MIPS}. On appelle ensuite la méthode \texttt{t->Fork(StartUserthead, (f, arg))}.

\item Dans \texttt{fork}:
On initialise la "partie noyau" du thread propulseur $t$ de manière
à ce qu'il exécute, la méthode \texttt{StartUserThread(f,arg)}. On affecte à $t$ le même espace d'adressage que le thread courant. On met à jour la liste des thread de l'espace d'adressage, on cherche un emplacement
libre dans l'espace d'adressage pour la pile en mode user du nouveau thread $t$.
On place ensuite le thread $t$ dans la ready list du scheduler.

\item Lorsque $t$ devient le thread courant il exécute la méthode \texttt{StartUserThread((f,arg))}
qui se charge de configurer les registres de la machine MIPS pour l'exécution
de la fonction utilisateur \texttt{f(arg)} et de lancer la machine avec \texttt{machine->Run()}. 
\end{enumerate}

\subsubsection{Création d'un processus}

\subsubsection{Terminaison d'un thread et d'un processus}
Les threads d'un processus ont une vie (et une mort) indépendante, sans aucune hiérarchie entre eux.
Un processus se termine en même temps que son dernier fil d'exécution.
L'absence de hiérarchie entre threads à l'intérieur d'un processus et entre processus simplifie
l'implémentation des méthodes \texttt{do\_UserThreadCreate} \texttt{do\_UserThreadexit}.

Une terminaison d'un thread (utilisateur) se fait par l'appel système \texttt{UserThreadFinish} qui appelle \texttt{do\_UserThreadexit}. Le thread courant consulte la \texttt{joinMap}, réveille les thread en attente. Enfin on appelle la méthode \texttt{finish} dans laquelle on met à jour la stackBitmap, la liste des threads de l'espace d'adressage, on décrémente le compteur des thread actifs sur le système. Le thread courant devient alors \texttt{threadTobeDestroyed} (ou éteint la machine si il était le dernier thread actif).

Au prochain appel de la méthode \texttt{run}, Si le thread à détruire était le dernier thread d'un espace d'adressage le scheduler appelle do\_UserThreadExit pour libérer les ressources et la mémoire associée au processus et détruit le thread propulseur.

\subsubsection{Terminaison automatique des threads}
Dans \texttt{Start.S} lors de l'appel système \texttt{UserThreadCreate}, on place dans le registre 6 de la machine mips l'adresse de l'instruction \texttt{UserThreadExit}.
Lors de la création du nouveau thread au niveau noyau avec \texttt{do\_UserThreadcreate} on récupère
cette adresse depuis \texttt{r6} et au moment de \texttt{Fork}, avant de brancher la machine \texttt{MIPS} sur
le nouveau thread utilisateur on place cette adresse dans le registre \texttt{RetAddrReg}. Ainsi au moment de terminer son exécution le \texttt{PC} sera branché sur UserThreadExit.

\subsubsection{Mémoire virtuelle}
La virtualisation de la mémoire des processus utilisateur est mise en oeuvre par une association entre numéro
de page en mémoire virtuelle et un numéro de page physique. Au moment de la création d'un espace d'adressage, l'allocation des pages physiques est réalisée par la méthode \texttt{GetEmptyframerandom()} de l'objet \texttt{FrameProvider}.
Dans cette méthode on construit, à l'aide de la \texttt{bitmap} de l'objet \texttt{frameprovider} une table temporaire des numéros de cadres de pages libres.
On choisit ensuite aléatoirement un numéro de cadre dans cette table. La construction de cette table est coûteuse en temps d'exécution mais facilite la mise en oeuvre du tirage aléatoire.

\subsection{Synchronisation}
Le système Nachos est exécute en tant que simple processus utilisateur, non multi-threadé. Aussi tout se passe comme si nous travaillions sur une architecture mono-processeur. Cela nous facilite la tâche pour l'implémentation des primitives de synchronisation au niveau noyau. En désactivant les interruptions on empêche toute tentative de préemption du scheduler. Pour les variables conditions 

\subsection{Réseau}
Nous avons mis en place un protocole de transmission de données fiable (sans perte de d'information), qui s'inspire du mécanisme d'acquittement cumulatif de TCP et de la notion de flux d'octet. On distingue deux type de messages, données et acquittements.
\subsubsection{Numéro de séquence}
On initie un flux d'octet bidirectionnel entre le serveur \texttt{A} et le client \texttt{B}.
Pour chaque direction de flux, chaque octet est identifié par un numéro de séquence unique. Chaque direction de flux possède un numéro de début de séquence initial et chaque octet du flux
possède un numéro de séquence relatif à ce numéro initial.
Lorsque l'on envoie un segment, on envoie un paquet d'octet d'un coup , le numéro de séquence transmit avec le segment
correspond au numéro de séquence du premier octet des données envoyées.

\subsection{Numéro d'acquittement}
L'émetteur d'un message d'acquittement transmet avec son message,
le numéro de séquence du prochain octet qu'il s'attend a recevoir.
Avec le principe d'acquittement cumulatif, \texttt{A} n'a pas besoin de recevoir les acquittements de tous les paquets qu'il envoie. Si on envoie trois paquets p1, p2, p3 à \texttt{B} et que \texttt{B}
acquitte p3, alors si \texttt{A} reçoit l'acquittement il sait que \texttt{B} a bien reçu p1, p2 et p3.

Contrairement à Tcp \texttt{B} envoie un acquittement pour chaque paquet reçu.

\subsubsection{Mise à jour des champs ACK et SEQ}
Lors d'échanges de paquets entre \texttt{A} et \texttt{B} on incrémente les valeurs des champs séquence et acquittement.
Pour simplifier notre protocole on exige que les envois se fassent dans l'ordre. Le numéro d' acquittement correspond au dernier octet reçu + $1$. Le numéro de séquence correspond à l'indice du premier octet du paquet envoyé.

On remarque que l'émetteur d'un acquittement ne sait pas si celui-ci
est bien arrivé à destination. Cela est lié au fait que la taille d'un paquet d'acquittement est nulle: on acquitte pas un acquittement. 

\subsubsection{Timer de rée-mission}
A chaque envoi de paquet de donnée on déclenche un timer associé à ce paquet.
En cas de perte du paquet, on attends un acquittement qui n'arrive pas, à l'écoulement du timer on rée-met le paquet, considéré perdu.
Il se peut aussi que les données arrivent mais que l'acquittement se perde, le résultat est le même. 

\subsubsection{ Envoi réception Données de tailles variables}
Lors de l'envoi d'un message dépassant \texttt{maxMailSize} on découpe le message en plusieurs paquets.

\subsection{Système de Fichier}

* Structures et variables importantes 
** FileOpened 
Lors de la création de l'objet \texttt{filesys} système de fichier on crée un tableau qui mémorise les fichiers ouvert. Ce tableau est utilisé pour garantir l'accès exclusif d'un fichier à un unique thread (sur tout le système). Les index dans le tableau sont utilisés comme descripteurs de fichiers fournis aux processus utilisateurs pour la lecture et l'écriture dans un fichier.


** Currentdirectory
A l'initialisation du système de fichier, on place lerepertoire courant à la racine du système de fichiers.

* Chemins relatifs
Toutes les méthodes createDirectory, changeDir, Create,... acceptent les chemins relatifs.

** FileHeaders
La gestion des fichiers et répertoires sur le disque s'appuie sur les objets fileHeader qui mémorise,
la taille du fichier en octet, le nombre de secteurs de disques utilisés, et les indices des secteurs de disques.
Le fileHeader est un index pour accéder au contenu réel du fichier.


Un répertoire est vu comme un fichier particulier et possède aussi son file header.
 
* FreeMap

* Is directory
Permet de discerner le type d' une entrée de répertoire (fichier ou répertoire).
Utilisé lors d'un change directory, et d'un remove (si le répertoire est vide ou pas).

\subsection{Autres Structures de données, variables globales du système}
\subsubsection{\texttt{thread/system.cc}}
\texttt{unsigned int threadCounter}: Nb total de thread Crées, sert à l'attribution d'un tid.

\texttt{unsigned int nbThreadActifs}: Nb Total de thread actifs du système, sert à éteindre la machine.

\texttt{unsigned int mutexCounter}: Nb Total de mutex créés, sert à l'initialisation des mutex utilisateur.

\texttt{std::map<int,Lock * > * mutexMap}: Association entre un identifiant de mutex (user) et un objet Lock.

\texttt{unsigned int semCounter}: Nb Total de sémaphore créées, sert à l'initialisation des sémaphores utilisateur.

\texttt{std::map<int,Semaphore * > * semMap}: Association entre un identifiant de sémaphore (user) et un objet Semaphore.

\texttt{unsigned int condCounter}: Nb Total de variables conditions crées, sert à l'initialisation d'une variable
condition utilisateur.

\texttt{std::map<int,Condition * > * condMap}: Association entre l'identifiant d'une variable condition et
un objet Condition.


\section{Organisation}
Pour les premières étapes nous avons travaillé sur un ordinateur commun.
Cette approche basée sur une communication constante et une implication de tous les membres du groupe
nous a permis d'échanger de manière constructive sur les choix d'implémentation, de s'assurer que
nous avions tous une compréhension et vision homogène du fonctionnement de notre code. Nous
avons adopté un point de vue critique vis à vis du code produit,
n'hésitant pas à discuter en détail des points qui ne nous semblaient pas clairs
avant pendant et après l'écriture de nos fonctions. Nous n'avons éprouvé aucune difficultés de communication
tant sur les aspects techniques que dans nos rapport au travail en général.

\section{Annexe}

\section{Tests}

\subsection{Entrées-Sorties}
\begin{itemize}
\item[-] \texttt{Putchar\_0}: Écriture du caractère \texttt{'a'} sur la sortie standard.
\item[-] \texttt{Putchar\_1}: Écriture du caractère \texttt{'a'} et \texttt{'b'} sur la sortie standard.
\item[-] \texttt{Putchar\_2}: Écriture de multiples caractères sur la sortie standard.
\item[-] \texttt{PutInt\_0}: Écriture de l' entier \texttt{10} sur la sortie standard.
\item[-] \texttt{PutInt\_1}: Écriture de l'entier \texttt{0} et de l'entier \texttt{1} sur la sortie standard.
\item[-] \texttt{GetInt\_0}: Lecture d'un entier depuis l'entrée standard.
\item[-] \texttt{GetInt\_PutInt\_0}:
  Lecture d'un entier depuis l'entrée standard.
  Écriture de cet entier sur la sortie standard.
\item[-] \texttt{GetString\_0}: Lecture d'une chaîne de moins de \texttt{20} caractères depuis l'entrée standard, affichage de cette chaîne sur la sortie standard.
\item[-] \texttt{PutString\_0}: Affichage de la chaîne de caractère \texttt{"ABCDEFGHIJklmnopqrstuvwxyz"}.
\item[-] \texttt{PutString\_1}: Affichage de la chaîne \texttt{"ABCDEFGH"} et \texttt{"ijklmnopqrstuvwxyz"}.
\item[-] \texttt{PutString\_2}:   Affichage d'une chaîne de taille supérieure à la
  constante \texttt{MAX\_STRING\_SIZE(=100)} définie dans \texttt{system.h}.
\end{itemize}

\subsection{UserThreads}
Afin de vérifier le bon fonctionnement des structures de synchronisation des fonctions d'entrée-sorties,
les programmes de tests ont été lancés avec et sans l'option \texttt{-rs} qui permet de rendre aléatoire le comportement de l'ordonnanceur.

\begin{itemize}
\item[-] \texttt{UserThreadcreate\_0}: Lancement de \texttt{N} threads utilisateurs qui affichent chacun un entier passé en paramètre lors de leur
  création.
\item[-] \texttt{Userthreadcreate\_1}: Lancement de deux threads utilisateur. Le premier affiche \texttt{"X"} puis le caractère \texttt{'a'} passé en paramètre.
  Le second afficher \texttt{"Y"} puis le caractère \texttt{'b'} passé en paramètre.
\item[-] \texttt{Userthreadcreate\_2}: Lancement d'un nombre de thread trop important pour l'espace mémoire.
  Le programme affiche l'identifiant des threads crées.
\item[-] \texttt{MultiThreadGetString\_0}: Lance plusieurs threads exécutant la commande \texttt{GetString()}.
\end{itemize}

\subsection{Entrées-Sorties Multithread}
 Avec les tests suivants, on vérifie le bon fonctionnement des structures de synchronisation par l'utilisation
  concurrente de synchconsole.
\begin{itemize}
\item[-] \texttt{MultithreadGetChar\_0}: Création de plusieurs threads exécutant \texttt{GetChar}
  puis \texttt{Putchar}.
  Remarque:
  A l'exécution, lorsqu'un thread s'exécute il se bloque sur l'instruction \texttt{GetChar()} et attends une
  entrée utilisateur. L'utilisateur entre un caractère \texttt{'c'} et \texttt{'$\backslash$n'} pour terminer son entrée.
  Suite à l'appel de \texttt{GetChar()}, le caractère \texttt{'$\backslash$n'} est toujours
  présent dans l'entrée standard. Aussi, le thread suivant exécutant \texttt{GetChar()} récupère et
  affiche \texttt{'$\backslash$n'}.
\item[-] \texttt{MultithreadGetInt\_0}: Création de plusieurs threads exécutant \texttt{GetInt}
  puis \texttt{PutInt} pour afficher l'entier saisi.
\item[-] \texttt{MultithreadGetString\_0}: Création de plusieurs threads exécutant
  \texttt{getString} puis \texttt{PutString}.
\item[-] \texttt{MultithreadPutChar\_0}: Création de plusieurs threads exécutant \texttt{getString}
  puis \texttt{PutString}.
\item[-] \texttt{MultithreadPutString\_0}: Création de plusieurs threads exécutant \texttt{PutString}.
\end{itemize}

\section{Consignes}
Un rapport de 8 à 12 pages maximum dont l'objectif est de nous convaincre d'acheter votre nachos. Ce rapport sera structuré en 5 parties :
une (courte) partie présentant rapidement les fonctionnalités intéressante/importante de votre noyau (ce qui vous démarque de vos concurrents, ce qu'on peut faire avec votre logiciel, ...).

Une partie "spécifications" listant ce qui est disponible pour les programmes utilisateurs.
Il faut mettre ici le genre d'information que vous trouvez dans les pages man.
On doit donc trouver tous les appels systèmes implémentés avec leur prototype, la description des arguments,
la description du fonctionnement (fonctionnalités utilisateurs, pas implémentation) de l'appel système,
de la valeur de retour éventuelle, la signalisation des erreurs, ...
Si vous avez également une bibliothèque utilisateur, vous devez décrire ses fonctions de la même manière que les appels systèmes.

Une partie "tests utilisateurs" décrivant les programmes de test que vous avez réalisés, ce qu'ils montrent, ...

Une partie "implémentation" qui explique les points importants de votre implémentation. C'est donc la seule partie qui parle du détail du code que vous avez écrit. Expliquez vos choix d'implémentation.
une partie plus "scolaire" où vous décrivez l'organisation de votre travail (planning, ...), commentaires constructifs sur le déroulement du projet, ...

Les 3 premières parties s'adressent à un utilisateur de votre système. La 4ième partie s'adresse à un développeur qui voudrait continuer à développer votre système. Si vous avez bien travaillé sur les spécifications comme demandé, ces parties techniques seront faciles à écrire.

\end{document}

