\documentclass[11pt]{article}
\usepackage[margin=1.6in]{geometry}
\usepackage[utf8]{inputenc}
\usepackage[T1]{fontenc}
\usepackage{fixltx2e}
\usepackage{graphicx}
\usepackage{longtable}
\usepackage{float}
\usepackage{wrapfig}
\usepackage{amsmath, amsthm}
\usepackage{textcomp}
\usepackage{marvosym}
\usepackage{wasysym}
\usepackage{amssymb}
\usepackage[frenchb]{babel}
\usepackage{color}
\usepackage{listings}
\lstset{
  %frame=tb,
  language=Bash,
  aboveskip=2mm,
  belowskip=2mm,
  showstringspaces=false,
  columns=flexible,
  basicstyle={\small\ttfamily},
  numbers=none,
  numberstyle=\footnotesize\color{gray},
  %keywordstyle=\color{blue},
  %commentstyle=\color{dkgreen},
  %stringstyle=\color{mauve},
  breaklines=true,
  breakatwhitespace=true,
  tabsize=3
}
         
% Pour XeTeX
\XeTeXdefaultencoding utf-8
\usepackage{fontspec}

\newenvironment{absolutelynopagebreak}
  {\par\nobreak\vfil\penalty0\vfilneg
   \vtop\bgroup}
  {\par\xdef\tpd{\the\prevdepth}\egroup
   \prevdepth=\tpd}


\definecolor{dkgreen}{rgb}{0,0.6,0}
\definecolor{gray}{rgb}{0.5,0.5,0.5}
\definecolor{mauve}{rgb}{0.58,0,0.82}

\theoremstyle{definition}
\newtheorem*{myRem}{Remarque}

\theoremstyle{definition}
\newtheorem*{myDef}{Définition}

\tolerance=1000
\setcounter{secnumdepth}{2}
\author{Borne, Duquennoy, Duverney, Isnel}
\date{}
\title{Nachos: Rapport final}

\begin{document}
\maketitle



\section{Fonctionnalités}
Environnement utilisateur multi-processus avec espaces d'adressages séparés et virtualisation mémoire.
Processus utilisateur multi-threadés avec primitives de synchronisation (Mutex, Semaphores, Variables-conditions). Gestion des entrées-sorties (\texttt{GetInt}, \texttt{PutInt}, \texttt{GetChar}, \texttt{PutChar}, \texttt{GetString}, \texttt{Putstring}).
Système de fichier. Protocole réseau sans connexion avec envoi fiable de données.
Utilisation interactive via un Shell (\texttt{exec}, \texttt{cd}, \texttt{ls}, \texttt{mkdir}, \texttt{rm},
\texttt{touch}, \texttt{cat}, \texttt{messagerie} ...).


\section{Appels système}
\subsection{Entrées-Sorties}

\subsubsection{\texttt{void PutChar(char c)}}
\begin{itemize}
\item[-] Sémantique: Écrit le caractère \texttt{c} sur la sortie standard.
\end{itemize}

\subsubsection{\texttt{void PutString(const char *s)}}
\begin{itemize}
\item[-] Sémantique: Écrit la chaîne de caractères lue à l'adresse \texttt{s} sur la sortie standard.
\item[-] Préconditions: La chaîne doit se terminer par un caractère \texttt{'$\backslash$0'}.
\end{itemize}

\subsubsection{\texttt{char GetChar()}}
\begin{itemize}
\item[-] Sémantique: Lit un caractère depuis l'entrée standard et retourne le caractère lu.
\end{itemize}

\subsubsection{\texttt{void GetString(char *s, int n)}}
\begin{itemize}
\item[-] Sémantique:
  Lit une chaîne de caractères de longueur maximale égale à \texttt{n} depuis l'entrée standard et
  l'écrit à l'adresse \texttt{s}.
\item[-] Préconditions: L'adresse \texttt{s} est valide (espace suffisant).
\end{itemize}

\subsubsection{\texttt{void PutInt(int n)}}
\begin{itemize}
\item[-] Sémantique: Écrit l'entier \texttt{n} sur la sortie standard.
\end{itemize}

\subsubsection{\texttt{void GetInt(int *n)}}
\begin{itemize}
\item[-] Sémantique: Lit un entier depuis l'entrée standard et l'écrit à l'adresse \texttt{n}.
\item[-] Préconditions: L'adresse \texttt{n} est valide.
\end{itemize}

\subsection{Processus et Threads}

\subsubsection{\texttt{int ForkExec(char * fileName)}}
 \begin{itemize}
 \item[-] Sémantique: Crée un nouveau processus qui exécute le fichier dont le nom est fourni en paramètre
\item[-] Préconditions: "fileName" est le nom d'un fichier exécutable au format \texttt{noff}.
 \end{itemize}

\subsubsection{\texttt{int UserThreadCreate(void f(void* arg), void* arg)}}
\begin{itemize}
\item[-] Spécifications: Prends en paramètres un pointeur de fonction "f" ne retournant pas de valeur
  et un pointeur "arg" vers le ou les paramètres de la fonction "f".
\item[-] Sémantique: Crée un nouveau thread utilisateur qui exécute la fonction \texttt{f(arg)}. 
\item[-] Préconditions: Le système doit disposer d'une quantité de mémoire suffisante pour allouer la
  pile du thread à créer. 
\item[-] Valeur de retour: retourne l'identificateur du thread créé, $-1$ si une erreur s'est produite lors de
  la création du thread.
\end{itemize}

\subsubsection{\texttt{void UserThreadExit()}}
\begin{itemize}
\item[-] Sémantique: Termine l'exécution du thread courant.
\end{itemize}

\subsubsection{\texttt{int UserThreadJoin(int tid)}}
\begin{itemize}
\item[-] Sémantique: Attend la terminaison du thread d'identifiant "tid", renvoie $-1$
  si le thread est déjà terminé, $0$ sinon.
\item[-] Préconditions: "tid" est un identifiant de thread valide, initialisé dans le processus courant
  à l'aide d'un appel à \texttt{UserThreadCreate}. Le thread courant n'a pas déjà fait \texttt{join(tid)}.
\end{itemize}

\subsection{Synchronisation}
\subsubsection{\texttt{Mutex\_t MutexCreate()}}
\begin{itemize}
\item[-]Sémantique: Initialise un mutex.
\item[-]Valeur de retour: Un identifiant de type \texttt{Mutex\_t} pour le mutex.
\end{itemize}

\subsubsection{\texttt{void MutexLock(Mutex\_t mutexId)}}
\begin{itemize}
\item[-]Sémantique: Acquiert le mutex d'identifiant "mutexId". Si "mutexId" est déverrouillé, il devient verrouillé
  et possédé par le thread appelant. Si le mutex d'identifiant "mutexId" est déjà verrouillé par un autre thread,
  le thread courant est suspendu jusqu'à ce que "mutexId" soit déverrouillé.
\item[-]Pré-Condition: "mutexId" est un identifiant de mutex valide, initialisé dans le processus
  courant par la méthode \texttt{MutexCreate}.
\end{itemize}

\subsubsection{\texttt{void MutexUnlock(Mutex\_t mutexId)}}
\begin{itemize}
\item[-]Sémantique: Relâche le mutex d'identifiant "mutexId".
\item[-]Précondition: "mutexId" est un identifiant de mutex valide, initialisé dans le processus
  courant par la méthode \texttt{MutexCreate}. "mutexId" est verrouillé.
\end{itemize}

\subsubsection{\texttt{MutexDestroy(Mutex\_t mutexId)}}
\begin{itemize}
\item[-]Sémantique: Détruit le mutex "mutexId".
\item[-]Préconditions: "mutexId" est un identifiant de mutex valide, initialisé dans le processus
  courant par la méthode \texttt{MutexCreate}.
  Le verrou est relâché. Détruire un verrou non relâché mène à un comportement
  non déterminé.
\end{itemize}

\subsubsection{\texttt{Sem\_t SemCreate(int initialValue)}}
\begin{itemize}
\item[-]Sémantique: Initialise une sémaphore avec la valeur "initialValue".
\item[-]Valeur de retour: Un identifiant de type \texttt{Sem\_t} pour la variable-condition.
\end{itemize}

\subsubsection{\texttt{void SemWait(Sem\_t semaphoreId)}}
\begin{itemize}
\item[-]Sémantique: Le thread courant attend que la sémaphore ait une valeur suppérieure à $0$
  et la décrémente.
\item[-]Précondition: "semaphoreId" est un identifiant de semaphore valide, initialisé dans le processus
  appellant par la méthode \texttt{SemCreate}.
\end{itemize}

\subsubsection{\texttt{void SemPost(Sem\_t semaphoreId)}}
\begin{itemize}
\item[-]Sémantique: Incrémente la valeur de la sémaphore, réveille un thread en attente de cette sémaphore
  si besoin.
\item[-]Préconditions: "semaphoreId" est un identifiant de semaphore valide, initialisé dans le processus
  courant par la méthode \texttt{SemCreate}.
\end{itemize}

\subsubsection{\texttt{void SemDestroy(Sem\_t semaphoreId)}}
\begin{itemize}
\item[-]Sémantique: Libère les ressources associées à la sémaphore.
\item[-]Préconditions: "semaphoreId" est un identifiant de semaphore valide, initialisé dans le processus
  courant par la méthode \texttt{SemCreate}. Aucun thread ne doit être en attente de la sémaphore.
\end{itemize}

\subsubsection{\texttt{Cond\_t CondCreate()}}
\begin{itemize}
\item[-]Sémantique: Initialise une variable-condition.
\item[-]Valeur de retour: Un identifiant de type \texttt{Cond\_t} pour la variable-condition.
\end{itemize}

\subsubsection{\texttt{void CondWait(Cond\_t condId, Mutex\_t mutedId)}}
\begin{itemize}
\item[-]Sémantique: Met le thread courant en sommeil dans la file d'attente associée à "condId" et
  relâche le verrou "mutexId".
\item[-]Préconditions: "mutexId" et "condId" sont des identifiants de mutex et variables-conditions
  valides, initialisés dans le processus courant par les méthodes \texttt{MutexCreate} et \texttt{CondCreate}.
  "mutexId" est verrouillé.
\end{itemize}

\subsubsection{\texttt{void CondSignal(Cond\_t condId)}}
\begin{itemize}
\item[-]Sémantique: Réveille un thread en sommeil dans la file d'attente associée à la variable-condition
  "condId". Si aucun thread n'est présent dans la liste, le signal est perdu.
\item[-]Préconditions: "contId" est un identifiant de variable-condition valide, initialisé dans le processus
  courant par la méthode \texttt{CondCreate}.
\end{itemize}

\subsubsection{\texttt{void CondBroadCast(Cond\_t condId)}}
\begin{itemize}
\item[-]Sémantique: Réveille tous les threads en sommeil dans la file d'attente associée à la
  variable-condition "condId".
\item[-]Préconditions: "contId" est un identifiant de variable-condition valide, initialisé dans le processus
  courant par la méthode \texttt{CondCreate}.
\end{itemize}

\subsubsection{\texttt{void CondDestroy(Cond\_t condId)}}
\begin{itemize}
\item[-]Sémantique: Libère les ressources associées à la variable condition "condId". 
\item[-]Préconditions: "contId" est un identifiant de variable-condition valide, initialisé dans le processus
  courant par la méthode \texttt{CondCreate}. Aucun thread n'est en attente dans la file associée à la variable.
\end{itemize}

\subsection{Système de fichier}

\subsubsection{\texttt{void Create (char *name, int initialSize)}}
\begin{itemize}
\item[-]Sémantique: Crée un fichier de nom "name" et de taille "initialSize".
\end{itemize}

\subsubsection{\texttt{OpenFileId Open(char *name)}}
\begin{itemize}
\item[-]Sémantique: Ouvre le fichier dont le nom est "name". 
\item[-]Valeur de retour: Retourne un descripteur de fichier de type \texttt{OpenFileId}
  permettant de lire et écrire dans le fichier ou $-1$ si l'ouverture à échoué.
\end{itemize}

\subsubsection{\texttt{int Write (char *buffer, int size, OpenFileId id)}}
\begin{itemize}
\item[-]Sémantique: Écrit "size" octet(s) depuis le fichier dont le descripteur est "id"
  dans le buffer "buffer".
\item[-]Préconditions: Le descripteur "id" doit être valide (fichier ouvert), initialisé dans le thread courant
  par la méthode \texttt{Open}. Tenter d'écrire dans un fichier non initialisé par open resulte en un comportement non spécifié.
\item[-]Valeur de retour: Nombre d'octets écrits dans le fichier.
\end{itemize}

\subsubsection{\texttt{int Read(char *buffer, int size, OpenFileId id)}}
\begin{itemize}
\item[-]Sémantique: Lit "size" octets depuis le fichier dont le descripteur est "id" dans le buffer
  dont l'adresse est "buffer". Si le fichier contient moins de "size" octets on lit tout les octets disponibles.
\item[-]Préconditions: Le descripteur doit être valide (fichier ouvert), initialisé dans le thread courant
  par la méthode \texttt{Open}.
  Tenter de lire dans un fichier non initialisé par \texttt{Open} résulte en un comportement non spécifié.
\item[-]Valeur de retour: Nombre d'octets lus.
\end{itemize}

\subsubsection{\texttt{void Close(OpenFileId id)}}
\begin{itemize}
\item[-]Sémantique: Ferme le fichier dont le descripteur est "id".
\item[-]Préconditions: Le descripteur doit être valide (fichier ouvert), initialisé dans le thread courant
  par la méthode \texttt{Open}. Tenter de fermer un fichier non initialisé par \texttt{Open} résulte en un comportement non spécifié.
\end{itemize}

\subsubsection{\texttt{void CreateDirectory(char * name)}}
\begin{itemize}
\item[-]Sémantique: Crée un répertoire dans le système de fichier Nachos, de nom "name"
  passé en paramètre.
\end{itemize}

\subsubsection{\texttt{void ChangeDirectoryPath(char * name)}}
\begin{itemize}
\item[-]Sémantique: Change le répertoire courant vers le répertoire de nom "name".
\item[-]Exemple: \texttt{ChangeDirectoryPath("./Dossier1/Dossier2")}
\end{itemize}

\subsubsection{\texttt{void ListDirectory(char * name)}}
\begin{itemize}
\item[-]Sémantique: Liste tout les fichiers et documents du répertoire dont le nom "name"
  est passé en paramètre.
\item[-]Exemple: \texttt{ListDirectory("./Dossier1/Dossier2")}
\end{itemize}

 \subsubsection{\texttt{int Remove(char * name)}}
\begin{itemize}
\item[-]Sémantique: Supprime le fichier ou répertoire dont le nom "name" est passé en paramètre
\item[-]Valeur de retour: $1$ si la suppression a réussi, $0$ sinon.
\item[-]Exemple: \texttt{Remove("./Dossier1/Dossier2")}
\end{itemize}

\subsection{Réseau}

\subsubsection{\texttt{void SendMessage(int addressDesti, int boxTo, int boxFrom, char * data)}}
\begin{itemize}
\item[-]Sémantique: Envoi du message "data" depuis la boite
 "boxfrom" vers la machine d'adresse "addressDesti" dans la boite "boxTo".
\item[-]Préconditions: La machine "addressDesti" doit être prête à recevoir des messages,
  i.e. avoir exécuté \texttt{ReceiveMessage}, les numéros de boites "boxTo" et "boxFrom" sont compris entre $0$ et $9$.
\end{itemize}

\subsubsection{\texttt{void ReceiveMessage(char * data, int box)}}
\begin{itemize}
\item[-]Sémantique: Initialise la réception d'un message depuis la boite "box".
  Les données reçues sont stockées à l'adresse "data".
\item[-]Préconditions: Le numéro de boite "box".
\end{itemize}


\section{Implémentation}

\subsection{Processus et Threads}

\begin{lstlisting}
Fichiers: userprog/userthread.cc, userprog/userprocess.cc, userprog/addrspace.cc, thread/thread.cc
\end{lstlisting}

Nous modélisons un processus par un objet \texttt{AddrSpace}.
Les fonctionnalités rendues par un objet \texttt{AddrSpace} dépassent la simple gestion de la mémoire puisque l'on trouve encapsulé au sein de cette classe, les méthodes relatives, entre autres, à la restauration et la sauvegarde de l'état du processeur de la machine \texttt{MIPS}.
Nous avons décidé d'étendre la sémantique de l'objet \texttt{addrSpace} en ajoutant une liste des threads actifs dans l'espace d'adressage, ainsi que les informations relatives aux éventuels appels à la méthode \texttt{join} entre ces threads.

\begin{myDef}
  On appellera \textit{espace d'adressage} une entité décrivant un espace mémoire, l'état de la machine \texttt{MIPS} et les informations relatives aux fils d'exécution actifs dans cet espace.
\end{myDef}
Afin de faire le lien entre les différents threads d'un même processus/espace d'adressage nous donnons à chaque objet \texttt{thread} une référence vers un objet \texttt{Addrspace}.
\begin{myRem}
Lors de notre réflexion sur l'implémentation des processus utilisateur, il nous semblait plus cohérent de déléguer à \texttt{addrspace} les méthodes de gestion de la mémoire et de créer une classe \texttt{processus} regroupant d'un côté un espace d'adressage et de l'autre les informations relatives aux fils d'exécutions vivant dans cet espace. Pour un petit gain de cohérence avons introduit une trop grande complexité de mise en oeuvre. Nous avons donc abandonné ce modèle au profit de celui décrit plus haut.
\end{myRem}

\subsubsection{Piles des threads}
Afin d'accueillir les piles de nouveaux threads on divise un espace d'adressage \texttt{addrSpace} en blocs de pages, de taille \texttt{NumPagesPerStack}.
Dans chaque espace on mémorise l'état courant des emplacement de piles grâce à la bitmap \texttt{stackBitmap}. 

\subsubsection{Thread Join}
Au sein d'un processus, un thread peut utiliser la méthode \texttt{UserThreadJoin} pour attendre la terminaison d'un autre thread.
On maintient dans chaque \texttt{addrSpace} une structure \texttt{joinMap}.
La map associe à un \texttt{tid} une liste de threads.

Lorsqu'un thread $t1$ fait un \texttt{join} sur un thread $t2$ on ajoute dans \texttt{joinMap}
l'association $(t2, [t1])$. Cette association signifie que $t2$ est attendu par $t1$.
Le thread $t1$ se met alors en sommeil et attends $t2$. 
Si $t3$ fait un \texttt{join} sur $t2$ la \texttt{joinMap} se retrouve dans l'état $(t2, [t1,t3])$ et $t3$ est mis en sommeil.
Lorsque $t2$ termine son exécution il place $t1$ et $t3$ dans la \texttt{readylist} du scheduler.

De plus, on maintient chaque \texttt{addrSpace} une liste \texttt{threadList} des threads actifs de l'espace d'adressage.
Un thread $t1$ ne peut faire un join que vers un thread $t2$ actif.
Un thread peut être attendu par plusieurs threads distincts.

\subsubsection{Création d'un thread}
Toute création d'un thread utilisateur s'effectue par l'appel système
\texttt{UserThreadCreate}
qui déclenche un passage en mode noyau via l'instruction \texttt{syscall}.
Le gestionnaire d'exception implémenté dans la méthode \texttt{ExceptionHandler} de la classe
\texttt{exception.cc} appelle alors la méthode \texttt{do\_UserThreadCreate()}.
L'initialisation du nouveau fil d'exécution s'effectue en trois étapes.

\begin{enumerate}
\item Dans \texttt{do\_UserThreadCreate}:

On crée un nouvel objet thread $t$ (thread noyau propulseur), destiné à configurer et brancher la machine \texttt{MIPS} sur l'exécution d'une nouvelle fonction $f$ au sein de l'espace d'adressage du thread/processus courant.
On commence par récupérer la fonction utilisateur $f$ à exécuter et ses arguments $arg$ dans les registres $4$ et $5$ de la machine \texttt{MIPS}. On appelle ensuite la méthode \texttt{t->Fork(StartUserthead, (f, arg))}.

\item Dans \texttt{Fork}:
On initialise la "partie noyau" du thread propulseur $t$ de manière
à ce qu'il exécute la méthode \texttt{StartUserThread(f,arg)}. On affecte à $t$ le même espace d'adressage que le thread courant. On met à jour la liste des threads de l'espace d'adressage, on cherche un emplacement
libre dans l'espace d'adressage pour la pile du nouveau thread $t$.
On place ensuite le thread $t$ dans la \texttt{readyList} du scheduler.

\item Lorsque $t$ devient le thread courant il exécute la méthode \texttt{StartUserThread(f,arg)}
qui se charge de configurer les registres de la machine \texttt{MIPS} pour l'exécution
de la fonction utilisateur \texttt{f(arg)} et de lancer la machine avec \texttt{machine->Run()}. 
\end{enumerate}

\subsubsection{Création d'un processus}
La procédure de création d'un processus est similaire à celle d'un thread utilisateur, à la différence que l'on doit lire un exécutable au format \texttt{noff} pour initialiser un nouvel espace d'adressage et affecter cet espace au thread propulseur avant l' appel à \texttt{Fork} dans la méthode \texttt{do_UserProcessCreate}.

\subsubsection{Terminaison d'un thread et d'un processus}
Les threads d'un processus ont une vie (et une mort) indépendante, sans aucune hiérarchie entre eux.
Un processus se termine en même temps que son dernier fil d'exécution.
L'absence de hiérarchie entre threads à l'intérieur d'un processus et entre processus simplifie
l'implémentation des méthodes \texttt{do\_UserThreadCreate} et \texttt{do\_UserThreadExit}.

Une terminaison d'un thread (utilisateur) se fait par l'appel système \texttt{UserThreadFinish} qui appelle \texttt{do\_UserThreadExit}. Le thread courant consulte la \texttt{joinMap}, réveille les thread en attente. Enfin on appelle la méthode \texttt{finish} dans laquelle on met à jour la stackBitmap, la liste des threads de l'espace d'adressage, on décrémente le compteur des threads actifs sur le système. Le thread courant devient alors \texttt{threadTobeDestroyed} (ou éteint la machine si il était le dernier thread actif).

Au prochain appel de la méthode \texttt{run}, si le thread à détruire était le dernier thread d'un espace d'adressage, le scheduler appelle \texttt{do\_UserThreadExit} pour libérer les ressources et la mémoire associée au processus et détruit le thread propulseur.

\subsubsection{Terminaison automatique des threads}
Dans \texttt{Start.S}, lors de l'appel système \texttt{UserThreadCreate}, on place dans le registre $6$ de la machine \texttt{MIPS} l'adresse de l'instruction \texttt{UserThreadExit}.
Lors de la création du nouveau thread au niveau noyau avec \texttt{do\_UserThreadcreate} on récupère
cette adresse depuis \texttt{r6} et au moment de \texttt{Fork}, avant de brancher la machine \texttt{MIPS} sur
le nouveau thread utilisateur, on place cette adresse dans le registre \texttt{RetAddrReg}. Ainsi au moment de terminer son exécution le \texttt{PC} sera branché sur UserThreadExit.

\subsubsection{Mémoire virtuelle}
La virtualisation de la mémoire des processus utilisateur est mise en oeuvre par une association entre numéro
de page en mémoire virtuelle et un numéro de page physique. Au moment de la création d'un espace d'adressage, l'allocation des pages physiques est réalisée par la méthode \texttt{GetEmptyframeRandom()} de l'objet \texttt{FrameProvider}.
Dans cette méthode on construit, à l'aide de la \texttt{bitmap} de l'objet \texttt{FrameProvider} une table temporaire des numéros de cadres de pages libres.
On choisit ensuite aléatoirement un numéro de cadre dans cette table. La construction de cette table est coûteuse en temps d'exécution mais facilite la mise en oeuvre du tirage aléatoire.

\subsection{Synchronisation}
L'implémentation des primitives de synchronisation au niveau utilisateur s'appuie sur les structures de données noyau suivantes que l'on trouvera dans la classe \texttt{system.cc}:
\begin{itemize}
\item[-]\texttt{unsigned int mutexCounter}: Nombre total de mutex créés,
  sert d'indentifiant lors de l'initialisation d'un mutex au niveau utilisateur.
\item[-]\texttt{std::map<int,Lock * > * mutexMap}: Association entre un identifiant de mutex
  et un objet Lock.
\item[-]\texttt{unsigned int semCounter}: Nombre total de sémaphores créées,
  sert d'identifiant lors de l'initialisation d'une sémaphores au niveau utilisateur.
\item[-]\texttt{std::map<int,Semaphore * > * semMap}: Association entre un identifiant de sémaphore
  et un objet Semaphore.
\item[-]\texttt{unsigned int condCounter}: Nombre total de variables conditions crées, sert
  d'identifiant lors de l'initialisation d'une variable condition utilisateur.
\item[-]\texttt{std::map<int,Condition * > * condMap}: Association entre l'identifiant d'une variable condition et un objet Condition.
\end{itemize}
nous avons ajouté à la "classe" usersynch.cc un ensemble de méthodes pour la création des objets mutex, sémaphores, variables-condition et la mise à jour des structures au niveau noyau servant de support à la synchronisation dans les processus utilisateurs.

\subsection{Réseau}
Nous avons mis en place un protocole simple de transmission fiable de données (sans perte de d'information), qui s'inspire du mécanisme d'acquittement cumulatif de TCP et de la notion de flux d'octet.
\begin{myDef}
  On appelle message, une séquence d'octets à transmettre sur le réseau.
\end{myDef}
L'envoi d'un message passe par l'envoi d'un ou plusieurs objets \texttt{mail} que l'on peut assimiler à un segment \texttt{TCP}. 

Un mail est constitué d'un en-tête et d'un tableau d'octets contenant les données à envoyer.
On distingue trois types de mail:
\begin{itemize}
\item[-]Donnée
\item[-]Acquittement
\item[-]Fin de message
\end{itemize}

On stocke dans l'en-tête d'un mail la taille des données, le type de mail, un numéro de séquence, un
numéro d'acquittement.

La fiabilité de notre protocole, tout comme \texttt{TCP} s'appuie sur l'utilisation de numéros de séquences
et de numéro d'acquittement.

\subsubsection{Numéro de séquence}
Supposons que l'on initie un flux d'octet bidirectionnel entre un serveur \texttt{A} et un client \texttt{B}.
Pour chaque direction de flux, tout octet du flux est identifié par un numéro de séquence unique. 
Lorsque l'on envoie un mail, on joint dans l'en-tête du mail un numéro de séquence correspondant
au numéro du premier octet des données envoyées dans ce mail. Le numéro de séquence initial est $0$.

\subsubsection{Numéro d'acquittement}
La fiabilité de la transmission repose sur le fait que, pour chaque mail envoyé par $A$ à $B$, $B$ prévient $A$ qu'il a bien reçu le mail en lui renvoyant un mail d'acquittement.
Un mail d'acquittement est un mail particulier, sans données, servant uniquement à transmettre un numéro d'acquittement correspondant numéro de séquence du prochain octet que $B$ s'attend a recevoir.
Avec le principe d'acquittement cumulatif, \texttt{A} n'a pas besoin de recevoir les acquittements de tous les paquets qu'il envoie. Si on envoie trois paquets p1, p2, p3 à \texttt{B} et que \texttt{B}
acquitte p3, alors si \texttt{A} reçoit l'acquittement il sait que \texttt{B} a bien reçu p1, p2 et p3.

\subsubsection{Timer de réemission}
Chaque mail, avant d'être envoyé, est encapsulé dans un paquet au niveau de la couche réseau (semblable à IP). A chaque envoi de paquet on déclenche un timer. A la "sonnerie" du timer, si on a pas reçu d'acquittement, on réemet le paquet considéré perdu. Il se peut aussi que les données arrivent mais que l'acquittement se perde, le résultat est le même. 

\subsubsection{Mise à jour des champs ACK et SEQ}
Lors d'échanges de paquets entre \texttt{A} et \texttt{B} on incrémente les valeurs des champs séquence et acquittement. Pour simplifier notre protocole on exige que les envois de données se fassent dans l'ordre. Le numéro d' acquittement correspond au dernier octet reçu + $1$. Le numéro de séquence correspond à l'indice du premier octet du paquet envoyé. L'émetteur d'un acquittement ne sait pas si celui-ci est bien arrivé à destination. On acquitte pas un acquittement. 

\subsubsection{ Envoi et réception messages de tailles variables}
Lors de l'envoi d'un message dépassant \texttt{maxMailSize} on découpe le message en plusieurs mails.
L'émetteur termine l'envoi du message par un mail de type "Fin de message" contenant pour seules données
la chaîne de caractères "fin". Une fois ce dernier mail acquitté, on est sûr que le message est bien arrivé.

\section{Système de Fichier}

\subsection{FileOpened}
Nous avons ajouté à l'objet \texttt{Filesys} un tableau \texttt{fileDescriptor* fileOpened[NBFILEOPENED]} qui mémorise les fichiers ouvert du système de fichier. Ce tableau est utilisé pour garantir l'accès exclusif d'un fichier à un unique thread sur tout le système. Les index dans le tableau servent de descripteurs de fichiers fournis aux processus utilisateurs pour la lecture et l'écriture dans un fichier.

\section{Currentdirectory}
A l'initialisation du système de fichier, on place le repertoire courant à la racine du système de fichiers.

\section{Chemins relatifs}
Toutes les méthodes createDirectory, changeDir, Create,... acceptent les chemins relatifs.

** FileHeaders
La gestion des fichiers et répertoires sur le disque s'appuie sur les objets \texttt{fileHeader} qui mémorisent:
\begin{itemize}
\item[-] La taille du fichier en octet
\item[-] Le nombre de secteurs de disques utilisés
\item[-] Les indices des secteurs de disques.
\end{itemize}
Le fileHeader est un index pour accéder au contenu réel du fichier.
un répertoire est vu comme un fichier particulier et possède aussi son file header.
 
* FreeMap

\section{Is directory}
Permet de discerner les fichiers des répertoire lorsqu'on parcourt les entrées d'un répertoire.
Utilisé lors d'un change directory, et d'un remove (si le répertoire est vide ou pas).

\section{Organisation}
Pour les premières étapes nous avons travaillé sur un ordinateur commun.
Cette approche basée sur une communication constante et une implication de tous les membres du groupe
nous a permis d'échanger de manière constructive sur les choix d'implémentation, de s'assurer que
nous avions tous une compréhension et vision homogène du fonctionnement de notre code. Nous
avons adopté un point de vue critique vis à vis du code produit,
n'hésitant pas à discuter en détail des points qui ne nous semblaient pas clairs
avant pendant et après l'écriture de nos fonctions. Nous n'avons éprouvé aucune difficultés de communication
tant sur les aspects techniques que dans nos rapport au travail en général.

\section{Pistes d'amélliorations}
Nous estimons que nous pourrions rapidement mettre en oeuvre les ammélioration suivantes:

Bibliothèque d'entrées sorties bufferisées, allocateur mémoire, nouvelles politiques pour le scheduler,
ouverture concurente de fichiers, taille variables des fichiers, taille maximum des fichiers.
Nous ne pouvons pas encore executer de manière interractive des programmes utilisateurs depuis l'invite de commande. Il nous faudrait pour cela implémenter un minimum de communication inter-processus (Signaux, waitPid).

\section{Annexe}

\section{Tests}

\subsection{Entrées-Sorties}
\begin{itemize}
\item[-] \texttt{Putchar\_0}: Écriture du caractère \texttt{'a'} sur la sortie standard.
\item[-] \texttt{Putchar\_1}: Écriture du caractère \texttt{'a'} et \texttt{'b'} sur la sortie standard.
\item[-] \texttt{Putchar\_2}: Écriture de multiples caractères sur la sortie standard.
\item[-] \texttt{PutInt\_0}: Écriture de l' entier \texttt{10} sur la sortie standard.
\item[-] \texttt{PutInt\_1}: Écriture de l'entier \texttt{0} et de l'entier \texttt{1} sur la sortie standard.
\item[-] \texttt{GetInt\_0}: Lecture d'un entier depuis l'entrée standard.
\item[-] \texttt{GetInt\_PutInt\_0}:
  Lecture d'un entier depuis l'entrée standard.
  Écriture de cet entier sur la sortie standard.
\item[-] \texttt{GetString\_0}: Lecture d'une chaîne de moins de \texttt{20} caractères depuis l'entrée standard, affichage de cette chaîne sur la sortie standard.
\item[-] \texttt{PutString\_0}: Affichage de la chaîne de caractère \texttt{"ABCDEFGHIJklmnopqrstuvwxyz"}.
\item[-] \texttt{PutString\_1}: Affichage de la chaîne \texttt{"ABCDEFGH"} et \texttt{"ijklmnopqrstuvwxyz"}.
\item[-] \texttt{PutString\_2}:   Affichage d'une chaîne de taille supérieure à la
  constante \texttt{MAX\_STRING\_SIZE(=100)} définie dans \texttt{system.h}.
\end{itemize}

\subsection{UserThreads}
Afin de vérifier le bon fonctionnement des structures de synchronisation des fonctions d'entrée-sorties,
les programmes de tests ont été lancés avec et sans l'option \texttt{-rs} qui permet de rendre aléatoire le comportement de l'ordonnanceur.

\begin{itemize}
\item[-] \texttt{UserThreadcreate\_0}: Lancement de \texttt{N} threads utilisateurs qui affichent chacun un entier passé en paramètre lors de leur
  création.
\item[-] \texttt{Userthreadcreate\_1}: Lancement de deux threads utilisateur. Le premier affiche \texttt{"X"} puis le caractère \texttt{'a'} passé en paramètre.
  Le second afficher \texttt{"Y"} puis le caractère \texttt{'b'} passé en paramètre.
\item[-] \texttt{Userthreadcreate\_2}: Lancement d'un nombre de thread trop important pour l'espace mémoire.
  Le programme affiche l'identifiant des threads crées.
\item[-] \texttt{MultiThreadGetString\_0}: Lance plusieurs threads exécutant la commande \texttt{GetString()}.
\end{itemize}

\subsection{Entrées-Sorties Multithread}
 Avec les tests suivants, on vérifie le bon fonctionnement des structures de synchronisation par l'utilisation
  concurrente de synchconsole.
\begin{itemize}
\item[-] \texttt{MultithreadGetChar\_0}: Création de plusieurs threads exécutant \texttt{GetChar}
  puis \texttt{Putchar}.
  Remarque:
  A l'exécution, lorsqu'un thread s'exécute il se bloque sur l'instruction \texttt{GetChar()} et attends une
  entrée utilisateur. L'utilisateur entre un caractère \texttt{'c'} et \texttt{'$\backslash$n'} pour terminer son entrée.
  Suite à l'appel de \texttt{GetChar()}, le caractère \texttt{'$\backslash$n'} est toujours
  présent dans l'entrée standard. Aussi, le thread suivant exécutant \texttt{GetChar()} récupère et
  affiche \texttt{'$\backslash$n'}.
\item[-] \texttt{MultithreadGetInt\_0}: Création de plusieurs threads exécutant \texttt{GetInt}
  puis \texttt{PutInt} pour afficher l'entier saisi.
\item[-] \texttt{MultithreadGetString\_0}: Création de plusieurs threads exécutant
  \texttt{getString} puis \texttt{PutString}.
\item[-] \texttt{MultithreadPutChar\_0}: Création de plusieurs threads exécutant \texttt{getString}
  puis \texttt{PutString}.
\item[-] \texttt{MultithreadPutString\_0}: Création de plusieurs threads exécutant \texttt{PutString}.
\end{itemize}

\section{Consignes}
Un rapport de 8 à 12 pages maximum dont l'objectif est de nous convaincre d'acheter votre nachos. Ce rapport sera structuré en 5 parties :
une (courte) partie présentant rapidement les fonctionnalités intéressante/importante de votre noyau (ce qui vous démarque de vos concurrents, ce qu'on peut faire avec votre logiciel, ...).

Une partie "spécifications" listant ce qui est disponible pour les programmes utilisateurs.
Il faut mettre ici le genre d'information que vous trouvez dans les pages man.
On doit donc trouver tous les appels systèmes implémentés avec leur prototype, la description des arguments,
la description du fonctionnement (fonctionnalités utilisateurs, pas implémentation) de l'appel système,
de la valeur de retour éventuelle, la signalisation des erreurs, ...
Si vous avez également une bibliothèque utilisateur, vous devez décrire ses fonctions de la même manière que les appels systèmes.

Une partie "tests utilisateurs" décrivant les programmes de test que vous avez réalisés, ce qu'ils montrent, ...

Une partie "implémentation" qui explique les points importants de votre implémentation. C'est donc la seule partie qui parle du détail du code que vous avez écrit. Expliquez vos choix d'implémentation.
une partie plus "scolaire" où vous décrivez l'organisation de votre travail (planning, ...), commentaires constructifs sur le déroulement du projet, ...

Les 3 premières parties s'adressent à un utilisateur de votre système. La 4ième partie s'adresse à un développeur qui voudrait continuer à développer votre système. Si vous avez bien travaillé sur les spécifications comme demandé, ces parties techniques seront faciles à écrire.

\end{document}

